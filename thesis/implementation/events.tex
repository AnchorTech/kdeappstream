Po stronie przeglądarki przechwytywane są wszystkie zdarzenia myszy oraz klawiatury. Każde zdarzenie jest zamieniane na obiekt \emph{JSON} i wysyłane do serwera przy użyciu \emph{Websocket}. Dodatkowo przesyłane są zdarzenia dotyczące manipulacji oknami -- zdarzenia zmiany rozmiary, zamknięcia oraz aktywacji okna. W ten sposób z poziomu przeglądarki możliwa jest całkowita kontrola aplikacji.

\subsection{Zdarzenia myszy}
W zdarzeniach myszy używane są następujące pola:
\begin{itemize}
\item type -- typ zdarzenia,
\item id -- identyfikator widgeta, którego dotyczy zdarzenie,
\item x -- pozycja na osi X w momencie zajścia zdarzenia,
\item y -- pozycja na osi Y w momencie zajścia zdarzenia,
\item ox -- poprzednia pozycja na osi X, używane tylko przy ruchu,
\item oy -- poprzednia pozycja na osi Y, używane tylko przy ruchu,
\item btn -- wartość liczbowa określająca przyciski wciśnięte w momencie zajścia zdarzenia,
\item modifiers -- wartość liczbowa określająca klawisze specjalne (Alt, Control, Shift, klawisz Windows, klawisz Menu) wciśnięte w momencie zajścia zdarzenia,
\item delta -- wartość przesunięcia, używane tylko przy przewijaniu,
\item orientation -- kierunek, używane tylko przy przewijaniu.
\end{itemize}

\subsubsection{Ruch myszy}

\begin{lstlisting}[language=JavaScript,numbers=none]
{
  "command": "mouse",
  "type": "move",
  "id": 123456,  // Identyfikator obiektu, ktorego dotyczy zdarzenie
  "x": 0.0,
  "y": 0.0,
  "ox": 0.0,
  "oy": 0.0,
  "btn": 0x0,    // 0x00000000 Qt::NoButton
                 // 0x00000001 Qt::LeftButton
                 // 0x00000002 Qt::RightButton
                 // 0x00000004 Qt::MiddleButton
                 // 0x00000008 Qt::XButton1
                 // 0x00000010 Qt::XButton2
  "modifiers": 0x0  // 0x00000000 Qt::NoModifier
                    // 0x02000000 Qt::ShiftModifier
                    // 0x04000000 Qt::ControlModifier
                    // 0x08000000 Qt::AltModifier
                    // 0x10000000 Qt::MetaModifier
                    // 0x20000000 Qt::KeypadModifier
                    // 0x40000000 Qt::GroupSwitchModifier
}
\end{lstlisting}

\subsubsection{Wciśnięcie przycisku myszy}
\begin{lstlisting}[language=JavaScript,numbers=none]
{
  "command": "mouse",
  "type": "press",
  "id": 123456,
  "x": 0.0,
  "y": 0.0,
  "btn": 0x00000001,       // Qt::LeftButton
  "modifiers": 0x00000000  //  Qt::NoModifier
}
\end{lstlisting}

\subsubsection{Zwolnienie przycisku myszy}
\begin{lstlisting}[language=JavaScript,numbers=none]
{
  "command": "mouse",
  "type": "release",
  "id": 123456,
  "x": 0.0,
  "y": 0.0,
  "btn": 0x00000001,       // Qt::LeftButton
  "modifiers": 0x02000000  //  Qt::ShiftModifier
}
\end{lstlisting}

\subsubsection{Podwójne kliknięcie}
\begin{lstlisting}[language=JavaScript,numbers=none]
{
  "command": "mouse",
  "type": "dbclick",
  "id": 123456,
  "x": 0.0,
  "y": 0.0,
  "btn": 0x00000002,       // Qt::RightButton
  "modifiers": 0x00000000  //  Qt::NoModifier
}
\end{lstlisting}

\subsubsection{Zmiana położenia kółka myszy}
\begin{lstlisting}[language=JavaScript,numbers=none]
{
  "command": "wheel",
  "id": 123456,
  "x": 0.0,
  "y": 0.0,
  "btn": 0x00000002,       // Qt::RightButton
  "modifiers": 0x00000000  //  Qt::NoModifier
  "delta": 120,        // Zmiana polozenia
  "orientation": 0x    // 0x1 Qt::Horizontal
                       // 0x2 Qt::Vertical
}
\end{lstlisting} 

\subsection{Zdarzenia klawiatury}
\subsubsection{Wciśnięcie klawisza}
\begin{lstlisting}[language=JavaScript,numbers=none]
{
  "command": "key",
  "type": "press", 
  "key": 0x193,    // Kod klawisza
  "text": "a",     // Ciag znakow UTF8 bedacy rezultatem zdarzenia
  "autorep": true|false, // Okresla czy zdarzenie jest wynikiem
                         // przytrzymania klawisza przed dluzszy czas
  "count": 0,    // Okresla ile powtorzen klawisza mialo miejsce
  "modifiers": 0x0  // 0x00000000 Qt::NoModifier
                    // 0x02000000 Qt::ShiftModifier
                    // 0x04000000 Qt::ControlModifier
                    // 0x08000000 Qt::AltModifier
                    // 0x10000000 Qt::MetaModifier
                    // 0x20000000 Qt::KeypadModifier
                    // 0x40000000 Qt::GroupSwitchModifier
}
\end{lstlisting}

\subsubsection{Zwolnienie klawisza}
\begin{lstlisting}[language=JavaScript,numbers=none]
{
  "command": "key",
  "type": "release", 
  "key": 0x193,    // Kod klawisza
  "text": "a",     // Ciag znakow UTF8 bedacy rezultatem zdarzenia
  "autorep": true|false, // Okresla czy zdarzenie jest wynikiem
                         // przytrzymania klawisza przed dluzszy czas
  "count": 0,    // Okresla ile powtorzen klawisza mialo miejsce
  "modifiers": 0x0  // Qt::NoModifier
}
\end{lstlisting}
 

\subsection{Zmiana rozmiaru okna}
\input{implementation/events/resize.tex}

\subsection{Aktywacja okna}
\begin{lstlisting}[language=JavaScript,numbers=none]
{
  "command": "activate",
  "id": 123456   // Identyfikator obiektu, ktorego dotyczy zdarzenie
}
\end{lstlisting} 

\subsection{Zamknięcie okna}
\begin{lstlisting}[language=JavaScript,numbers=none]
{
  "command": "close",
  "id": 123456,   // Identyfikator obiektu, ktorego dotyczy zdarzenie
}
\end{lstlisting} 
