
W tym podrozdziale zostaną przedstawione mechanizmy biblioteki Qt wykorzystane przy tworzeniu projektu, o którym stanowi niniejsza praca. 

\subsection{System zdarzeń}
W \emph{Qt} zdarzenia są obiektami dziedziczącymi po klasie \emph{QEvent}, reprezentującymi zajście pewnego zjawiska wewnątrz aplikacji lub będącymi wynikiem oddziaływania z zewnątrz, o którym aplikacja powinna wiedzieć. Zdarzenia mogą być przetworzone przez wszystkie obiekty dziedziczące po klasie \emph{QObject}, która dostarcza podstawowej struktury i logiki niezbędnej do ich obsługi. 

Kiedy system operacyjny generuje sygnał o zajściu pewnego zdarzenia, \emph{Qt} dokonuje jego konwersji na odpowiedni i platformowo niezależny format. Każde zdarzenie jest następnie przekazywane do \emph{kolejki zdarzeń} odpowiedniego wątku. Kolejka przechowuje i w odpowiednim momencie rozdysponowywuje zdarzenia do odpowiadających im obiektów odbiorców poprzez wywołanie metody \emph{QObject::event()} wewnątrz której następuje decyzja dotycząca dalszego przetwarzania, zależna od rodzaju zdarzenia. 

Niektóre zdarzenia, takie jak na przykład \emph{QMouseEvent} czy \emph{QKeyEvent} pochodzą bezpośrednio od systemu operacyjnego. Inne, jak na przykład \emph{QTimerEvent} czy \emph{QPaintEvent} pochodzą z innych źródeł, nierzadko z wnętrza samej aplikacji (np. do komunikacji między wątkami). Warto w tym miejscu zaznaczyć, że rysowanie w \emph{Qt} nie jest operacją wywoływaną przez system operacyjny lecz przez samą aplikację oraz rysowanie z wnętrza obsługi zdarzenia \emph{QPaintEvent} jest jedynym sposobem na renderowanie graficznego interfejsu aplikacji. Pociąga to za sobą pewne problemy opisane w dalszej części pracy.

\subsection{System widgetów}
Widget'em w bibliotece \emph{Qt} nazywamy obiekt reprezentujący elementy graficznego interfejsu użytkownika takie jak przyciski, listy rozwijane, menu, okna i inne. Klasa \emph{QWidget}\footnote{http://doc.qt.digia.com/qt/qwidget.html} ta jest typem bazowym dla wszystkich widgetów i udostępnia niezbędne metody dotyczące renderowania oraz obsługi zdarzeń dzięki czemu w łatwy sposób mozemy uzyskać dostęp do całego interfejsu aplikacji.

Interfejs użytkownika w aplikacjach opartych o framework Qt tworzy strukturę hierarchiczną powiązanych ze sobą obiektów klasy QWidget. Wykorzystując ten fakt w łatwy sposób można odtworzyć tą strukturę w innych technologiach, np. tworząc identyczną strukturę w języku HTML. Fakt ten został wykorzystany w niniejszej pracy.

\subsection{System rysowania}
\label{system_rysowania}
Rysowanie w bibliotece Qt standardowo zostało zaimplementowane dla rysowania na ekranie oraz urządzeniach drukujących wykorzystując natywne API systemu operacyjnego, dla którego dana wersja Qt została skompilowana. Moduł ten jest niejako opakowaniem dla wywołań systemowych, ujednolicając jego logikę i umożliwiając pełną przenośność aplikacji. Na rysunku \ref{paintsystem-core} przedstawiony został kaskadowy model systemu rysowania w Qt. Jest to \emph{model trójwarstwowy} i każda z klas ma swoje określone zadanie w całym procesie renderowania. Główną zaletą takiego podejścia jest ujednolicenie przepływu procesu rysowania dla różnych urządzeń wyjściowych oraz umożliwienie łatwego sposobu dla dodawania nowych funkcjonalności.

Klasa \emph{QPainter} udostępnia jednolity interfejs umożliwjający wykonywanie operacji rysowania różnych obiektów takich jak linie, okręgi, prostokąty, obrazy oraz umożliwia zastosowanie różnego rodzaju przekształceń, styli czy transformacji macierzowych. 

Klasa \emph{QPaintDevice} stanowi abstrakcję dla dwuwymiarowej przestrzeni na której obiekty klasy \emph{QPainter} mogą wykonywać operacje rysowania. Udostępnia ona różnego rodzaju informacje dotyczące specyfiki urządzenia wyjściowego, które mogą być wykorzystane np. do optymalizacji procesu rysowania. 

Klasa \emph{QPaintEngine} udostępnia interfejs, za pomocą którego obiekty klasy \emph{QPainter} będą mogły wykonywać operacje rysowania na różnego rodzaju urządzeniach wyjściowych. Klasa \emph{QPaintEngine} jest używana wewnątrz klas \emph{QPainter} oraz \emph{QPaintDevice} i jest ukryta przed aplikacjiami dopóki programista nie zechce stworzyć obsługi dla nowego rodzaju urządzenia wyjściowego. W niniejszej pracy taki właśnie scenariusz został wykorzystany.
 
\begin{figure}[!h]
  \centering
  \includegraphics[width=\textwidth,height=!]{img/paintsystem-core.png}
  \caption{Schemat budowy systemu renderowania w bibliotece Qt}
  \label{paintsystem-core}
\end{figure}