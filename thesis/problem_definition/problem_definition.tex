Przedmiotem pracy jest stworzenie prototypowego serwera oraz klienta \emph{HTML5}. Zadaniem serwera jest jest udostępnienie usługi uruchamiającej aplikacje oparte o biblioteki \emph{Qt} i \emph{KDE}. Zdalny dostęp do aplikacji uruchomionej w środowisku serwera jest dostępny poprzez interfejs \emph{HTML5} udostępniany przez instancję serwera.

Użytkownik serwisu w celu skorzystania z aplikacji \emph{Qt} zainstalowanej na serwerze wchodzi na odpowiedni adres przy użyciu nowoczesnej przeglądarki internetowej. Następnie wybiera interesujący go program i przejmuje nad nim kontrolę. Jest w stanie wyświetlić program w przeglądarce, który wizualnie odpowiada rzeczywistej instancji uruchomionej na serwerze. Wszystkie akcje myszy oraz klawiatury przechwycone przez przeglądarke są wysyłane do serwera, dzięki czemu użytkownik ma pełną kontrolę nad uruchomioną aplikacją. Projekt zakłąda również implementację odpowiednika menadżera okien (ang. window manager) po stronie klienta, aby udostępnić użytkownikowi funkcjonalność pracy z wieloma oknami --- przesuwanie, rozszerzanie, zamykanie, oraz opcjonalnie minimalizowanie oraz maksymalizowanie.

Postawione zadanie w głównej mierze polega na rozwiązaniu czterech podstawowych problemów:
\begin{enumerate}
  \item komunikacja między klientem a serwerem,
  \item komunikacja między klientem a aplikacją,
  \item uzyskanie informacji o wyglądzie elementów graficznego interfejsu aplikacji,
  \item symulacja interakcji użytkownika z interfejsem aplikacji.
\end{enumerate}

Na rysunku \ref{fig:arch} przedstawiono ideowy schemat architektury systemu rozwiązującego powyższe kwestie. 
Podstawą projektu jest jego modułowość, która separuje logikę odpowiedzialną za udostępnianie interfejsu WWW inicjującego proces aplikacji \emph{Qt} od części stanowiącej węzeł komunikacyjny pomiędzy klientem a aplikacją \emph{Qt} działąjącą po stronie serwera.
Można również zauważyć bardzo wyraźne rozgraniczenie między dwoma kanałami przesyłu danych, które wynika z konieczności umożliwienia korzystania z serwera wielu klientom. W dalszej części przedstawiono opisy poszczególnych modułów oraz przepływów danych między różnymi częściami architektury na wyższym poziomie szczegółowości.

\begin{figure}[H]
\centering
\includegraphics[width=1.0\linewidth]{img/arch}
\caption{Schemat architektury systemu.}
\label{fig:arch}
\end{figure}

\section{Komunikacja między klientem a serwerem}
%Specyfiką problemu jest jego dwuetapowość. W pierwszym etapipe klient inicjuje połączenie jednorazowo wysyłając zapytanie zawierające informacje o aplikacji, którą klient chce uruchomić oraz identyfikatorze klienta. W drugiej kolejności wymagane jest utworzenie kanału komunikacyjnego między klientem a procesem aplikacji. 

Do realizacji zadania komunikacji między klientem a serwerem stworzony został prosty serwer WWW działający w oparciu o protokół \emph{HTTP}. Jego architektura została przedstawiona na rysunku \ref{fig:arch-www}. Jako zasób domyślny udostępnia on listę dostępnych aplikacji, które klient może uruchomić. Inicjalizacja połączenia polega na wysłaniu przez klienta identyfikatora wybranej aplikacji. Serwer po pomyślnej weryfikacji przydziela klientowi unikatowy identyfikator sesji, uruchamia proces aplikacji i wysyła klientowi skrypt w języku \emph{JavaScript} zajmujący się przetwarzaniem po stronie przeglądarki. Każde zapytanie klienta jest weryfikowane za pomocą modułu \emph{ACL}\footnote{ang. Access Control List}, który sprawdza czy konfiguracja serwera zezwala na uruchamianie żądanych aplikacji. Szczegółowy opis modułu przedstawiono w sekcji \ref{sec:server-security}.

\begin{figure}[H]
\centering
\includegraphics[width=1.0\linewidth]{img/arch-www}
\caption{Schemat komunikacji z serwerem WWW.}
\label{fig:arch-www}
\end{figure}

\section{Komunikacja między klientem a aplikacją}

Do rozwiązania tego problemu konieczne jest utworzenie ciągłego kanału komunikacyjnego między klientem a procesem aplikacji, za pomocą którego będzie możliwe przesyłanie informacji o wyglądzie interfejsu aplikacji oraz informowanie aplikacji o zdarzeniach generowanych przez użytkownika po stronie przeglądarki. Jako, że za cel przyjęte zostało założenie o nieingerowaniu bezpośrednio w kod skompilowanych już aplikacji, głównym założeniem jest skorzystanie z techniki wstrzykiwania kodu biblioteki dynamicznej do przestrzeni pamięciowej procesu aplikacji tuż przed jego uruchomieniem. Kod ten ma za zadanie utrzymanie połączenia oraz transmisję danych między klientem a aplikacją. Dokładny sposób użycia tej techniki jest przedstawiony w sekcji \ref{sec:ldpreload}.

\begin{figure}[H]
\centering
\includegraphics[width=1.0\linewidth]{img/arch-socket}
\caption{Schemat komunikacji z modułem WebSocket.}
\label{fig:arch-socket}
\end{figure}

\begin{figure}[H]
\centering
\includegraphics[width=1.0\linewidth]{img/arch-hook}
\caption{Schemat powiązań modułu WebSocket z aplikacją \emph{Qt/KDE}.}
\label{fig:arch-hook}
\end{figure}

\section{Uzyskanie informacji o wyglądzie elementów graficznego interfejsu aplikacji}

Każdy element graficznego interfejsu aplikacji (\emph{QWidget}) jest renderowany w momencie odebrania zdarzenia \emph{QPaintEvent} z kolejki zdarzeń głównego wątku aplikacji. Dzięki temu istnieje łatwy sposób na uzyskanie informacji o tym kiedy oraz ktory element należy przerenderować aby uaktualnić jego wygląd po stronie klienta. Problemem w dalszyb ciągu pozostaje jednak sposób na uzyskanie informacji o samym wyglądzie. 

Proponowane rozwiązanie polega na zaimplementowaniu abstrakcyjnego urządzenia wyjściowego reprezentującego przeglądarkę WWW po stronie klienta (patrz podrozdział \ref{system_rysowania}). Odpowiednio implementując klasy \emph{QPaintEngine} oraz \emph{QPaintDevice} możliwe staje się uzyskanie szczegółowych informacji dotyczących wygądu widgetów co z kolei umozliwia stworzenie innowacyjnego formatu przesyłanych danych. Jego innowacyjność polega na przerzuceniu odpowiedzialności za rysowanie elementów piksel po pikselu na stronę klienta na podstawie dostarczonych przez serwer niezbędnych do tego celu informacji. Zamiast przesyłać bitmapy z wyrenderowanym elementem można wysłać informacje o kolorach, punktach, liniach i innych podstawowych elementach, które zostaną narysowane na urządzeniu docelowym jakim po stronie klienta jest przeglądarka WWW z obsługą elementów \emph{canvas}. Na tle istniejących rozwiązań jest to podejście dotąd niespotykane.


\section{Symulacja interakcji użytkownika z interfejsem aplikacji}
Interakcja użytkownika z aplikacją sprowadza się do obsługi następujących zdarzeń:
\begin{enumerate}
  \item ruch myszy nad elementem,
  \item wciśnięcie, zwolnienie oraz dwuklik przycisku myszy,
  \item zmiana położenia kółka myszy,
  \item wciśnięcie oraz zwolnienie klawiszy na klawiaturze,
  \item zmiana rozmiaru okna aplikacji poprzez przeciąganie jego krawędzi,
  \item zamknięcie, minimalizacja lub maksymalizacja okna aplikacji.
\end{enumerate}
Większość z wyżej wymienionych elementów jest obsługiwana jako zdarzenia w języku JavaScript większości dzisiejszych przeglądarek. Proponowane podejście na rozwiązanie tego zagadnienia polega na stworzeniu formatu danych bazując na notacji JSON (JavaScript Object Notation). Dane w tym formacie przesyłane do serwera są następnie poddawane walidacji i konwersji na obiekty zdarzeń biblioteki \emph{Qt}. Zdarzenia takie są następnie przesyłane do kolejki zdarzeń w głównym wątku aplikacji.

Odbiorcą zdarzenia jest obiekt, który na rysunku \ref{fig:arch-hook} został oznaczony jako Event Dispather. Moduł ten jest następnie odpowiedzialny za podjęcie akcji w odpowiedzi na dane zdarzenie, polegających na wywołaniu odpowiednich metod na elemencie interfejsu graficznego aplikacji, który wygenerował dane zdarzenie po stronie przeglądarki bazując na hierarchicznej budowie interfejsu użytkownika.
Wyjątkami są tutaj zdarzenia klawiatury, które nie mają bezpośredniego odbiorcy w momencie ich zaistnienia. Aplikacja na ogół sama decyduje o tym, który element powinien odebrać zdarzenie. Domyślnie jest to widget, który atualnie posiada tzw. focus, a to z kolei zależy od poprzednich zdarzeń oraz logiki samego programu. W celu symulacji podobnego zachowania moduł Event Dispather wybiera element interfejsu bazując na aktualnym stanie aplikacji i wysyła informację o wciśnięciu wirtualnych klawiszy. 

Taka struktura modułu obsługi zdarzeń zapewnia jednolitość działania aplikacji pomiędzy różnymi wersjami biblioteki Qt.

\section{Odtwarzanie aplikacji po stronie klienta}

\begin{figure}[H]
\centering
\includegraphics[width=0.8\linewidth]{img/arch-render}
\caption{Schemat architektury odtwarzania wyglądu aplikacji po stronie klienta.}
\label{fig:arch-render}
\end{figure}


\section{Wewnętrzne przepływy danych w module WebSocket serwera}

\begin{figure}
\centering
\includegraphics[width=0.8\linewidth]{img/arch-lib}
\caption{Schemat wewnętrznych przepływów danych w module WebSocket.}
\label{fig:arch-lib}
\end{figure}


\section{Zabezpieczenie serwera}
\label{sec:server-security}
Ponieważ jednym z celów projektu było umożliwienie uruchamiania pełnoprawnych aplikacji zainstalowanych na systemie operacyjnym serwera, kluczową staje się możliwość blokowania nieautoryzowanego dostępu do wrażliwych lub potencjalnie niebezpiecznych aplikacji.

W związku z powyższym, stworzono mechanizm list \emph{ACL} (ang. Access Control Lists), który pozwala administratorowi systemu na zdefiniowane, które aplikacje mogą być uruchamiane przez klientów, a w przypadku których zostanie wyświetlony komunikatu o braku dostępu.

Listy kontroli dostępu przechowywane są w pliku konfiguracyjnym serwera w postaci danych w formacie \emph{XML}\footnote{ang. Extensible Markup Language}. Nazwy aplikacji w postaci komend linii poleceń mogą więc być definiowane ręcznie w dowolnym edytorze tekstowym lub za pomocą pliku wykonywalnego serwera poprzez poniższe argumentów wywołania programu:

\begin{itemize}
\item \emph{accept-all}
spowoduje zniesienie wszystkich wcześniej wprowadzonych obostrzeń i możliwe będzie uruchomienie wszystkich aplikacji zainstalowanych na serwerze,
\item \emph{reject-all}
spowoduje zablokowanie wszystkich zapytań serwera. Komenda ta powinna stanowić pierwszy krok w etapie budowy list dostępu,
\item \emph{accept nazwa-aplikacji} \footnote{Nazwa aplikacji oznacza pełną komendę wiersza poleceń (wraz z możliwymi argumentami), która spowoduje uruchomienie aplikacji. Może to być również ścieżka bezwzględna do pliku wykonywalnego aplikacji.}
spowoduje, że aplikacja o podanej nazwie będzie mogła być uruchamiana przez serwer,
\item \emph{reject nazwa-aplikacji},
komenda blokujaca możliwość uruchamiania aplikacji o podanej nazwie.
\end{itemize}

Serwer, ze względów bezpieczeństwa, od razu po zainstalowaniu domyslnie blokuje wszystkie zapytania klientów i oczekuje się od administratora serwera skonfigurowania list \emph{ACL} według uznania. Zmiana ustawień serwera wymaga jego ponownego uruchomienia.
