Przedmiotem pracy jest stworzenie prototypowego serwera hostującego desktopowe aplikacjie oparte o biblioteki Qt oraz KDE. Na żądanie klienta serwer uruchamia wybraną aplikację oraz wstrzykuje kod odpowiedzialny za komunikację klienta z procesem aplikacji i przesyłanie klientowi danych dotyczących wyglądu graficznego interfejsu aplikacji. 

Postawione zadanie w głównej mierze polega na rozwiązaniu trzech podstawowych problemów:
\begin{enumerate}
  \item Komunikacja między klientem a serwerem
  \item Komunikacja między klientem a aplikacją
  \item Uzyskanie informacji o wyglądzie elementów graficznego interfejsu aplikacji
  \item Symulacja interakcji użytkownika z interfejsem aplikacji
\end{enumerate}