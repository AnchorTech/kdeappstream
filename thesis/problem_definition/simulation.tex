Interakcja użytkownika z aplikacją sprowadza się do obsługi następujących zdarzeń:
\begin{enumerate}
  \item ruch myszy nad elementem,
  \item wciśnięcie, zwolnienie oraz dwuklik przycisku myszy,
  \item zmiana położenia kółka myszy,
  \item wciśnięcie oraz zwolnienie klawiszy na klawiaturze,
  \item zmiana rozmiaru okna aplikacji poprzez przeciąganie jego krawędzi,
  \item zamknięcie, minimalizacja lub maksymalizacja okna aplikacji.
\end{enumerate}
Większość z wyżej wymienionych elementów jest obsługiwana jako zdarzenia w języku JavaScript większości dzisiejszych przeglądarek. Proponowane podejście na rozwiązanie tego zagadnienia polega na stworzeniu formatu danych bazując na notacji JSON (JavaScript Object Notation). Dane w tym formacie przesyłane do serwera są następnie poddawane walidacji i konwersji na obiekty zdarzeń biblioteki \emph{Qt}. Zdarzenia takie są następnie przesyłane do kolejki zdarzeń w głównym wątku aplikacji.

Odbiorcą zdarzenia jest obiekt, który na rysunku \ref{fig:arch-hook} został oznaczony jako Event Dispather. Moduł ten jest następnie odpowiedzialny za podjęcie akcji w odpowiedzi na dane zdarzenie, polegających na wywołaniu odpowiednich metod na elemencie interfejsu graficznego aplikacji, który wygenerował dane zdarzenie po stronie przeglądarki bazując na hierarchicznej budowie interfejsu użytkownika.
Wyjątkami są tutaj zdarzenia klawiatury, które nie mają bezpośredniego odbiorcy w momencie ich zaistnienia. Aplikacja na ogół sama decyduje o tym, który element powinien odebrać zdarzenie. Domyślnie jest to widget, który atualnie posiada tzw. focus, a to z kolei zależy od poprzednich zdarzeń oraz logiki samego programu. W celu symulacji podobnego zachowania moduł Event Dispather wybiera element interfejsu bazując na aktualnym stanie aplikacji i wysyła informację o wciśnięciu wirtualnych klawiszy. 

Taka struktura modułu obsługi zdarzeń zapewnia jednolitość działania aplikacji pomiędzy różnymi wersjami biblioteki Qt.