Ponieważ jednym z celów projektu było umożliwienie uruchamiania pełnoprawnych aplikacji zainstalowanych na systemie operacyjnym serwera, kluczową staje się możliwość blokowania nieautoryzowanego dostępu do wrażliwych lub potencjalnie niebezpiecznych aplikacji.

W związku z powyższym, stworzono mechanizm list \emph{ACL} (ang. Access Control Lists), który pozwala administratorowi systemu na zdefiniowane, które aplikacje mogą być uruchamiane przez klientów, a w przypadku których zostanie wyświetlony komunikatu o braku dostępu.

Listy kontroli dostępu przechowywane są w pliku konfiguracyjnym serwera w postaci danych w formacie \emph{XML}\footnote{ang. Extensible Markup Language}. Nazwy aplikacji w postaci komend linii poleceń mogą więc być definiowane ręcznie w dowolnym edytorze tekstowym lub za pomocą pliku wykonywalnego serwera poprzez poniższe argumentów wywołania programu:

\begin{itemize}
\item \emph{accept-all}
spowoduje zniesienie wszystkich wcześniej wprowadzonych obostrzeń i możliwe będzie uruchomienie wszystkich aplikacji zainstalowanych na serwerze,
\item \emph{reject-all}
spowoduje zablokowanie wszystkich zapytań serwera. Komenda ta powinna stanowić pierwszy krok w etapie budowy list dostępu,
\item \emph{accept nazwa-aplikacji} \footnote{Nazwa aplikacji oznacza pełną komendę wiersza poleceń (wraz z możliwymi argumentami), która spowoduje uruchomienie aplikacji. Może to być również ścieżka bezwzględna do pliku wykonywalnego aplikacji.}
spowoduje, że aplikacja o podanej nazwie będzie mogła być uruchamiana przez serwer,
\item \emph{reject nazwa-aplikacji},
komenda blokujaca możliwość uruchamiania aplikacji o podanej nazwie.
\end{itemize}

Serwer, ze względów bezpieczeństwa, od razu po zainstalowaniu domyslnie blokuje wszystkie zapytania klientów i oczekuje się od administratora serwera skonfigurowania list \emph{ACL} według uznania. Zmiana ustawień serwera wymaga jego ponownego uruchomienia.