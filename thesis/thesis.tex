\documentclass[polish]{inz}

%+Make Index
\usepackage{makeidx}
\makeindex
%-Make Index

\usepackage{polski}
\usepackage[utf8]{inputenc}
\usepackage[OT4]{fontenc}
\usepackage{listings}
\usepackage{color}
\usepackage[usenames,dvipsnames]{xcolor}

%+Colors definitions
\definecolor{white}{rgb}{1,1,1}
\definecolor{lightgray}{rgb}{.9,.9,.9}
\definecolor{darkgray}{rgb}{.4,.4,.4}
\definecolor{purple}{rgb}{0.65, 0.12, 0.82}
%-Colors definitions

\lstdefinelanguage{JavaScript}{
  keywords={typeof, new, true, false, catch, function, return, null, catch, switch, var, if, in, while, do, else, case, break},
  keywordstyle=\color{blue}\bfseries,
  ndkeywords={class, export, boolean, throw, implements, import, this},
  ndkeywordstyle=\color{darkgray}\bfseries,
  identifierstyle=\color{black},
  sensitive=false,
  comment=[l]{//},
  morecomment=[s]{/*}{*/},
  commentstyle=\color{purple}\ttfamily,
  stringstyle=\color{red}\ttfamily,
  morestring=[b]',
  morestring=[b]"
}

\lstset{
   language=C++,
   emph={QPaintEngine,Qt,QTextItem,QRect,QRectF,QPoint,QPointF,QLine,QLineF,QEllipse,QPainterPath,QPixmap,QImage},
   emphstyle={\color{RedViolet}\bfseries}
}

\lstset{
   language=JavaScript,
   backgroundcolor=\color{white},
   extendedchars=true,
   basicstyle=\footnotesize\ttfamily,
   showstringspaces=false,
   showspaces=false,
   numbers=left,
   numberstyle=\footnotesize,
   numbersep=9pt,
   tabsize=2,
   breaklines=true,
   showtabs=false,
   captionpos=b
}

%+Title
\title{Graficzne interfejsy aplikacji opartych o biblioteki Qt i KDE}
\author{Jan Jędrychowski\\Łukasz Spas}
\date{2012}
\advisor{dr inż. Igor Wojnicki}
%-Title

\begin{document}
\maketitle

\chapter{Wstęp}

W dzisiejszych czasach coraz bardziej powszechne staje się wykorzystanie przeglądarek do zadań, do których wcześniej używane były duże aplikacje klienckie. Powstają rozwiązania, które starają się oddzielić logikę obliczeniową od warstwy prezentacji, przenosząc jednocześnie tę pierwszą na stronę serwera. Rozwój technologii HTML5 rozszerzającej standard o elementy canvas, websocket, webworkers i inne umożliwia tworzenie aplikacji o możliwościach takich samych jakie niegdyś były dostępne tylko w programach desktopowych. Co więcej gwarantuje międzyplatformowość nie tylko w rozumieniu softwareowym -- jedna aplikacja dostępna jest zarówno na komputerach osobistych, tabletach, telefonach i innych urządzeniach wyposażonych w nowoczesną przeglądarkę. Przy użyciu bardzo związanej z HTML5 technologii CSS3 możliwie jest tworzenie jednej aplikacji, która będzie użytkowalna niezależnie od wielkości ekranu urządzenia.

W niektórych rozwiązaniach zastąpienie starych aplikacji desktopowych nowymi aplikacjami webowymi (przeglądarkowymi) jest jednak niemożliwe, czasochłonne lub zbyt kosztowne.

Podczas badań rynku pod kątem aktualnie dostępnych technologii dostrzeżono braki w rozwiązaniach umożliwiających zdalną interakcję z pojedynczymi aplikacjami. Większość z rozwiązań dostępnych na rynku wymusza udostępnienie całego pulpitu oraz wymaga od użytkownika końcowego (klienta) posiadania odpowiedniego, nierzadko płatnego oprogramowania (np. TeamViewer, VNC, Citrix, X11 i inne). Celem projektu jest stworzenie alternatywy wymagającej od strony klienta jedynie przeglądarki obsługującej HTML5 bez konieczności instalacji jakichkolwiek pluginów (np. Java, Flash).

Głównym wzorcem dla tej pracy jest projekt GTK+ Broadway powstały w 2011 roku oferujący dostęp przez przeglądarkę internetową do aplikacji działających pod kontrolą biblioteki GTK na zdalnym serwerze. Do tej pory nie istniało rozwiązanie oferujące podobną funkcjonalność dla biblioteki Qt i stworzony na potrzeby tej pracy projekt jest pierwszą taką implementacją. Kluczowym czynnikiem wyróżniającym tę pracę na tle innych jest innowacyjny sposób przesyłu danych do wizualizacji okien i ich elementów, który nie opieraja się na transmisji bitmap. Zamiast pełnych bitmap przesyłane są informacje sterującę rysowaniem pochodzące z silnika renderującego QT.
Takie rozwiązanie wykorzysta pełne możliwości elementu \emph{canvas} z HTML5 i pozwoli na przetestowanie praktycznej implementacji większości dostępnych funkcjonalności.

// TODO - streszczenie kolejnych rozdzialow?

\chapter{Podstawy teoretyczne}
W rodziale tym przedstawione zostaną najważniejsze informacje dotyczące technologii wykorzystanych w projekcie. 

\section{Wybrane rozwiązania HTML5}
\emph{HTML5} (ang. HyperText Markup Language) jest najnowszą wersją popularnego języka znaczników \emph{HTML}. Pojęcie to nie jest do końca jasne i oczywiste, ponieważ ta edycja języka niesie ze sobą nie tylko zmiany w znacznikach, ale bardzo mocno rozszerza możliwości stron WWW. Co więcej łączy się bezpośrednio z innymi technologiami takimi jak \emph{JavaScript} oraz \emph{CSS3} i nie jest w stanie bez nich istnieć. W związku z tym sama definicja \emph{HTML5} jako jedynie język znaczników jest niepełna. We wcześniejszych etapach samo konsorcjum W3 miało problemy z jasną definicją \emph{HTML5} i na krótki czas składowymi tej technologii był język \emph{CSS3} oraz \emph{SVG}. \cite{html5games}

Grupa \emph{W3} opublikowała 17 grudnia pierwszą oficjalną specyfikację \emph{HTML5} \cite{html5w3}. Od tej pory do standardu nie będą dodawane nowe funkcjonalności, a jedynie poprawiane aktualnie istniejące. Wydanie rekomendacji przez instytucję jest planowane na rok 2014 \cite{plan2014}.

\emph{HTML5} jest rozwijany w ścisłej współpracy z twórcami najpopularniejszych przeglądarek. Została powołana specjalna grupa \emph{WHATWG} (Web Hypertext Application Technology Working Group), która skupia producentów takich jak Mozilla Foundation, Google, Opera Software oraz Apple Inc. Przeglądarki internetowe takie jak Mozilla Firefox, Google Chrome oraz Opera już teraz implementują w wersjach produkcyjnych większość z planowanych nowości przedstawionych w aktualnej specyfikacji \emph{HTML5}.

W proponowanym rozwiązaniu, po stronie klienta zaadoptowano dwa nowe komponenty \emph{HTML5}: canvas (ang. płótno) oraz WebSocket.

\subsection{Element canvas}
Nowy element drzewa \emph{DOM} \emph{canvas} pozwala na renderowanie dynamicznych bitmap na stronie przy pomocy skryptów języka \emph{JavaScript}. Aktualnie wszystkie przeglądarki producentów z \emph{WHATWG} implementują obecny standard.
Wprowadzenie tego komponentu pozwala na tworzenie dowolnych animacji oraz grafik, których użycie wcześniej wymagało użycia zewnętrznych pluginów (np. \emph{Flash} lub \emph{Java}).
W projekcie element ten używany jest do rysowania pojedynczych widgetów.

\subsection{Technologia \emph{WebSocket}}
\emph{WebSocket} jest technologią oferującą ustandaryzowaną, pełną, dwustronną komunikację między klientem (przeglądarką internetową) a serwerem. Podobną funkcjonalność można było wcześniej zasymulować przy pomocy modelu \emph{Comet} korzystającego z długotrwałych połączeń \emph{HTTP}, na które leniwie (ang. lazy) były wysyłane dane. Poprzednie rozwiązanie z powodu braku ustandaryzowania oraz wykorzystywania obejścia było trudne w utrzymaniu oraz nie oferowało komunikacji dwustronnej.
W projekcie technologia wykorzystywana jest do łączności z serwerem.


\section{Opis biblioteki Qt}

W tym podrozdziale zostaną przedstawione mechanizmy biblioteki Qt wykorzystane przy tworzeniu projektu, o którym stanowi niniejsza praca. 

\subsection{System zdarzeń}
W \emph{Qt} zdarzenia są obiektami dziedziczącymi po klasie \emph{QEvent}, reprezentującymi zajście pewnego zjawiska wewnątrz aplikacji lub będącymi wynikiem oddziaływania z zewnątrz, o którym aplikacja powinna wiedzieć. Zdarzenia mogą być przetworzone przez wszystkie obiekty dziedziczące po klasie \emph{QObject}, która dostarcza podstawowej struktury i logiki niezbędnej do ich obsługi. 

Kiedy system operacyjny generuje sygnał o zajściu pewnego zdarzenia, \emph{Qt} dokonuje jego konwersji na odpowiedni i platformowo niezależny format. Każde zdarzenie jest następnie przekazywane do \emph{kolejki zdarzeń} odpowiedniego wątku. Kolejka przechowuje i w odpowiednim momencie rozdysponowywuje zdarzenia do odpowiadających im obiektów odbiorców poprzez wywołanie metody \emph{QObject::event()} wewnątrz której następuje decyzja dotycząca dalszego przetwarzania, zależna od rodzaju zdarzenia. 

Niektóre zdarzenia, takie jak na przykład \emph{QMouseEvent} czy \emph{QKeyEvent} pochodzą bezpośrednio od systemu operacyjnego. Inne, jak na przykład \emph{QTimerEvent} czy \emph{QPaintEvent} pochodzą z innych źródeł, nierzadko z wnętrza samej aplikacji (np. do komunikacji między wątkami). Warto w tym miejscu zaznaczyć, że rysowanie w \emph{Qt} nie jest operacją wywoływaną przez system operacyjny lecz przez samą aplikację oraz rysowanie z wnętrza obsługi zdarzenia \emph{QPaintEvent} jest jedynym sposobem na renderowanie graficznego interfejsu aplikacji. Pociąga to za sobą pewne problemy opisane w dalszej części pracy.

\subsection{System widgetów}
Widget'em w bibliotece \emph{Qt} nazywamy obiekt reprezentujący elementy graficznego interfejsu użytkownika takie jak przyciski, listy rozwijane, menu, okna i inne. Klasa \emph{QWidget}\footnote{http://doc.qt.digia.com/qt/qwidget.html} ta jest typem bazowym dla wszystkich widgetów i udostępnia niezbędne metody dotyczące renderowania oraz obsługi zdarzeń dzięki czemu w łatwy sposób mozemy uzyskać dostęp do całego interfejsu aplikacji.

Interfejs użytkownika w aplikacjach opartych o framework Qt tworzy strukturę hierarchiczną powiązanych ze sobą obiektów klasy QWidget. Wykorzystując ten fakt w łatwy sposób można odtworzyć tą strukturę w innych technologiach, np. tworząc identyczną strukturę w języku HTML. Fakt ten został wykorzystany w niniejszej pracy.

\subsection{System rysowania}
\label{system_rysowania}
Rysowanie w bibliotece Qt standardowo zostało zaimplementowane dla rysowania na ekranie oraz urządzeniach drukujących wykorzystując natywne API systemu operacyjnego, dla którego dana wersja Qt została skompilowana. Moduł ten jest niejako opakowaniem dla wywołań systemowych, ujednolicając jego logikę i umożliwiając pełną przenośność aplikacji. Na rysunku \ref{paintsystem-core} przedstawiony został kaskadowy model systemu rysowania w Qt. Jest to \emph{model trójwarstwowy} i każda z klas ma swoje określone zadanie w całym procesie renderowania. Główną zaletą takiego podejścia jest ujednolicenie przepływu procesu rysowania dla różnych urządzeń wyjściowych oraz umożliwienie łatwego sposobu dla dodawania nowych funkcjonalności.

Klasa \emph{QPainter} udostępnia jednolity interfejs umożliwjający wykonywanie operacji rysowania różnych obiektów takich jak linie, okręgi, prostokąty, obrazy oraz umożliwia zastosowanie różnego rodzaju przekształceń, styli czy transformacji macierzowych. 

Klasa \emph{QPaintDevice} stanowi abstrakcję dla dwuwymiarowej przestrzeni na której obiekty klasy \emph{QPainter} mogą wykonywać operacje rysowania. Udostępnia ona różnego rodzaju informacje dotyczące specyfiki urządzenia wyjściowego, które mogą być wykorzystane np. do optymalizacji procesu rysowania. 

Klasa \emph{QPaintEngine} udostępnia interfejs, za pomocą którego obiekty klasy \emph{QPainter} będą mogły wykonywać operacje rysowania na różnego rodzaju urządzeniach wyjściowych. Klasa \emph{QPaintEngine} jest używana wewnątrz klas \emph{QPainter} oraz \emph{QPaintDevice} i jest ukryta przed aplikacjiami dopóki programista nie zechce stworzyć obsługi dla nowego rodzaju urządzenia wyjściowego. W niniejszej pracy taki właśnie scenariusz został wykorzystany.
 
\begin{figure}[!h]
  \centering
  \includegraphics[width=\textwidth,height=!]{img/paintsystem-core.png}
  \caption{Schemat budowy systemu renderowania w bibliotece Qt}
  \label{paintsystem-core}
\end{figure}

\chapter{Określenie problemu i proponowane rozwiązanie}

Przedmiotem pracy jest stworzenie prototypowego serwera oraz klienta HTML5. Zadaniem serwera jest jest udostępnienie usługi uruchamiającej aplikacje oparte o biblioteki \emph{Qt} i KDE. Zdalny dostęp do aplikacji uruchomionej w środowisku serwera jest dostępny poprzez interfejs HTML5 udostępniany przez instancję serwera.

Użytkownik serwisu w celu skorzystania z aplikacji \emph{Qt} zainstalowanej na serwerze wchodzi na odpowiedni adres przy użyciu nowoczesnej przeglądarki internetowej. Następnie wybiera interesujący go program i przejmuje nad nim kontrolę. Jest w stanie wyświetlić program w przeglądarce, który wizualnie odpowiada rzeczywistej instancji uruchomionej na serwerze. Wszystkie akcje myszy oraz klawiatury przechwycone przez przeglądarke są wysyłane do serwera, dzięki czemu użytkownik ma pełną kontrolę nad uruchomioną aplikacją. Należy również zaimplementować odpowiednik menedżera okien (ang. window manager) po stronie klienta, aby udostępnić użytkownikowi funkcjonalność pracy z wieloma oknami --- przesuwanie, rozszerzanie, zamykanie, oraz opcjonalnie minimalizowanie oraz maksymalizowanie.

// TODO: Wymagania (ogólne)
// TODO: Przypadki użycia
// TODO: Zachowanie systemu / software'u
// TODO: Struktura systemu / software'u

Postawione zadanie w głównej mierze polega na rozwiązaniu czterech podstawowych problemów:
\begin{enumerate}
  \item komunikacja między klientem a serwerem,
  \item komunikacja między klientem a aplikacją,
  \item uzyskanie informacji o wyglądzie elementów graficznego interfejsu aplikacji,
  \item symulacja interakcji użytkownika z interfejsem aplikacji.
\end{enumerate}

Na rysunku \ref{fig:arch} przedstawiono ideowy schemat architektury systemu rozwiązującego powyższe kwestie. 
Podstawą projektu jest jego modułowość, która separuje logikę odpowiedzialną za udostępnianie interfejsu WWW inicjującego proces aplikacji Qt od części stanowiącej węzeł komunikacyjny pomiędzy klientem a aplikacją Qt działąjącą po stronie serwera.
Można również zauważyć bardzo wyraźne rozgraniczenie między dwoma kanałami przesyłu danych, które wynika z konieczności umożliwienia korzystania z serwera wielu klientom. 
W dalszej części przedstawiono opisy poszczególnych modułów oraz przepływów danych między różnymi częściami architektury na większym poziomie szczegółowości.

\begin{figure}[H]
\centering
\includegraphics[width=1.0\linewidth]{img/arch}
\caption{Schemat architektury systemu.}
\label{fig:arch}
\end{figure}

\section{Komunikacja między klientem a serwerem}
%Specyfiką problemu jest jego dwuetapowość. W pierwszym etapipe klient inicjuje połączenie jednorazowo wysyłając zapytanie zawierające informacje o aplikacji, którą klient chce uruchomić oraz identyfikatorze klienta. W drugiej kolejności wymagane jest utworzenie kanału komunikacyjnego między klientem a procesem aplikacji. 

Do realizacji zadania komunikacji między klientem a serwerem stworzony został prosty serwer WWW działający w oparciu o protokół \emph{HTTP}. Jego architektura została przedstawiona na rysunku \ref{fig:arch-www}. Jako zasób domyślny udostępnia on listę dostępnych aplikacji, które klient może uruchomić. Inicjalizacja połączenia polega na wysłaniu przez klienta identyfikatora wybranej aplikacji. Serwer po pomyślnej weryfikacji przydziela klientowi unikatowy identyfikator sesji, uruchamia proces aplikacji i wysyła klientowi skrypt w języku \emph{JavaScript} zajmujący się przetwarzaniem po stronie przeglądarki. Każde zapytanie klienta jest weryfikowane za pomocą modułu \emph{ACL}\footnote{ang. Access Control List}, który sprawdza czy konfiguracja serwera zezwala na uruchamianie żądanych aplikacji. Szczegółowy opis modułu przedstawiono w sekcji \ref{sec:server-security}.

\begin{figure}[H]
\centering
\includegraphics[width=1.0\linewidth]{img/arch-www}
\caption{Schemat komunikacji z serwerem WWW.}
\label{fig:arch-www}
\end{figure}

\section{Komunikacja między klientem a aplikacją}

Do rozwiązania tego problemu konieczne jest utworzenie ciągłego kanału komunikacyjnego między klientem a procesem aplikacji, za pomocą którego będzie możliwe przesyłanie informacji o wyglądzie interfejsu aplikacji oraz informowanie aplikacji o zdarzeniach generowanych przez użytkownika po stronie przeglądarki. Jako, że za cel przyjęte zostało założenie o nieingerowaniu bezpośrednio w kod skompilowanych już aplikacji, głównym założeniem jest skorzystanie z techniki wstrzykiwania kodu biblioteki dynamicznej do przestrzeni pamięciowej procesu aplikacji tuż przed jego uruchomieniem. Kod ten ma za zadanie utrzymanie połączenia oraz transmisję danych między klientem a aplikacją. Dokładny sposób użycia tej techniki jest przedstawiony w sekcji \ref{sec:ldpreload}.

\begin{figure}[H]
\centering
\includegraphics[width=1.0\linewidth]{img/arch-socket}
\caption{Schemat komunikacji z modułem WebSocket.}
\label{fig:arch-socket}
\end{figure}

\begin{figure}[H]
\centering
\includegraphics[width=1.0\linewidth]{img/arch-hook}
\caption{Schemat powiązań modułu WebSocket z aplikacją \emph{Qt/KDE}.}
\label{fig:arch-hook}
\end{figure}

\section{Uzyskanie informacji o wyglądzie elementów graficznego interfejsu aplikacji}

Każdy element graficznego interfejsu aplikacji (\emph{QWidget}) jest renderowany w momencie odebrania zdarzenia \emph{QPaintEvent} z kolejki zdarzeń głównego wątku aplikacji. Dzięki temu istnieje łatwy sposób na uzyskanie informacji o tym kiedy oraz ktory element należy przerenderować aby uaktualnić jego wygląd po stronie klienta. Problemem w dalszyb ciągu pozostaje jednak sposób na uzyskanie informacji o samym wyglądzie. 

Proponowane rozwiązanie polega na zaimplementowaniu abstrakcyjnego urządzenia wyjściowego reprezentującego przeglądarkę WWW po stronie klienta (patrz podrozdział \ref{system_rysowania}). Odpowiednio implementując klasy \emph{QPaintEngine} oraz \emph{QPaintDevice} możliwe staje się uzyskanie szczegółowych informacji dotyczących wygądu widgetów co z kolei umozliwia stworzenie innowacyjnego formatu przesyłanych danych. Jego innowacyjność polega na przerzuceniu odpowiedzialności za rysowanie elementów piksel po pikselu na stronę klienta na podstawie dostarczonych przez serwer niezbędnych do tego celu informacji. Zamiast przesyłać bitmapy z wyrenderowanym elementem można wysłać informacje o kolorach, punktach, liniach i innych podstawowych elementach, które zostaną narysowane na urządzeniu docelowym jakim po stronie klienta jest przeglądarka WWW z obsługą elementów \emph{canvas}. Na tle istniejących rozwiązań jest to podejście dotąd niespotykane.


\section{Symulacja interakcji użytkownika z interfejsem aplikacji}
Interakcja użytkownika z aplikacją sprowadza się do obsługi następujących zdarzeń:
\begin{enumerate}
  \item ruch myszy nad elementem,
  \item wciśnięcie, zwolnienie oraz dwuklik przycisku myszy,
  \item zmiana położenia kółka myszy,
  \item wciśnięcie oraz zwolnienie klawiszy na klawiaturze,
  \item zmiana rozmiaru okna aplikacji poprzez przeciąganie jego krawędzi,
  \item zamknięcie, minimalizacja lub maksymalizacja okna aplikacji.
\end{enumerate}
Większość z wyżej wymienionych elementów jest obsługiwana jako zdarzenia w języku JavaScript większości dzisiejszych przeglądarek. Proponowane podejście na rozwiązanie tego zagadnienia polega na stworzeniu formatu danych bazując na notacji JSON (JavaScript Object Notation). Dane w tym formacie przesyłane do serwera są następnie poddawane walidacji i konwersji na obiekty zdarzeń biblioteki \emph{Qt}. Zdarzenia takie są następnie przesyłane do kolejki zdarzeń w głównym wątku aplikacji.

Odbiorcą zdarzenia jest obiekt, który na rysunku \ref{fig:arch-hook} został oznaczony jako Event Dispather. Moduł ten jest następnie odpowiedzialny za podjęcie akcji w odpowiedzi na dane zdarzenie, polegających na wywołaniu odpowiednich metod na elemencie interfejsu graficznego aplikacji, który wygenerował dane zdarzenie po stronie przeglądarki bazując na hierarchicznej budowie interfejsu użytkownika.
Wyjątkami są tutaj zdarzenia klawiatury, które nie mają bezpośredniego odbiorcy w momencie ich zaistnienia. Aplikacja na ogół sama decyduje o tym, który element powinien odebrać zdarzenie. Domyślnie jest to widget, który atualnie posiada tzw. focus, a to z kolei zależy od poprzednich zdarzeń oraz logiki samego programu. W celu symulacji podobnego zachowania moduł Event Dispather wybiera element interfejsu bazując na aktualnym stanie aplikacji i wysyła informację o wciśnięciu wirtualnych klawiszy. 

Taka struktura modułu obsługi zdarzeń zapewnia jednolitość działania aplikacji pomiędzy różnymi wersjami biblioteki Qt.

\section{Odtwarzanie aplikacji po stronie klienta}

\begin{figure}[H]
\centering
\includegraphics[width=0.8\linewidth]{img/arch-render}
\caption{Schemat architektury odtwarzania wyglądu aplikacji po stronie klienta.}
\label{fig:arch-render}
\end{figure}


\section{Wewnętrzne przepływy danych w module WebSocket serwera}

\begin{figure}
\centering
\includegraphics[width=0.8\linewidth]{img/arch-lib}
\caption{Schemat wewnętrznych przepływów danych w module WebSocket.}
\label{fig:arch-lib}
\end{figure}


\section{Zabezpieczenie serwera}
\label{sec:server-security}
Ponieważ jednym z celów projektu było umożliwienie uruchamiania pełnoprawnych aplikacji zainstalowanych na systemie operacyjnym serwera, kluczową staje się możliwość blokowania nieautoryzowanego dostępu do wrażliwych lub potencjalnie niebezpiecznych aplikacji.

W związku z powyższym, stworzono mechanizm list \emph{ACL} (ang. Access Control Lists), który pozwala administratorowi systemu na zdefiniowane, które aplikacje mogą być uruchamiane przez klientów, a w przypadku których zostanie wyświetlony komunikatu o braku dostępu.

Listy kontroli dostępu przechowywane są w pliku konfiguracyjnym serwera w postaci danych w formacie \emph{XML}\footnote{ang. Extensible Markup Language}. Nazwy aplikacji w postaci komend linii poleceń mogą więc być definiowane ręcznie w dowolnym edytorze tekstowym lub za pomocą pliku wykonywalnego serwera poprzez poniższe argumentów wywołania programu:

\begin{itemize}
\item \emph{accept-all}
spowoduje zniesienie wszystkich wcześniej wprowadzonych obostrzeń i możliwe będzie uruchomienie wszystkich aplikacji zainstalowanych na serwerze,
\item \emph{reject-all}
spowoduje zablokowanie wszystkich zapytań serwera. Komenda ta powinna stanowić pierwszy krok w etapie budowy list dostępu,
\item \emph{accept nazwa-aplikacji} \footnote{Nazwa aplikacji oznacza pełną komendę wiersza poleceń (wraz z możliwymi argumentami), która spowoduje uruchomienie aplikacji. Może to być również ścieżka bezwzględna do pliku wykonywalnego aplikacji.}
spowoduje, że aplikacja o podanej nazwie będzie mogła być uruchamiana przez serwer,
\item \emph{reject nazwa-aplikacji},
komenda blokujaca możliwość uruchamiania aplikacji o podanej nazwie.
\end{itemize}

Serwer, ze względów bezpieczeństwa, od razu po zainstalowaniu domyslnie blokuje wszystkie zapytania klientów i oczekuje się od administratora serwera skonfigurowania list \emph{ACL} według uznania. Zmiana ustawień serwera wymaga jego ponownego uruchomienia.


\section{Komunikacja między klientem a serwerem}
%Specyfiką problemu jest jego dwuetapowość. W pierwszym etapipe klient inicjuje połączenie jednorazowo wysyłając zapytanie zawierające informacje o aplikacji, którą klient chce uruchomić oraz identyfikatorze klienta. W drugiej kolejności wymagane jest utworzenie kanału komunikacyjnego między klientem a procesem aplikacji. 

Do realizacji zadania komunikacji między klientem a serwerem stworzony został prosty serwer WWW działający w oparciu o protokół \emph{HTTP}. Jego architektura została przedstawiona na rysunku \ref{fig:arch-www}. Jako zasób domyślny udostępnia on listę dostępnych aplikacji, które klient może uruchomić. Inicjalizacja połączenia polega na wysłaniu przez klienta identyfikatora wybranej aplikacji. Serwer po pomyślnej weryfikacji przydziela klientowi unikatowy identyfikator sesji, uruchamia proces aplikacji i wysyła klientowi skrypt w języku \emph{JavaScript} zajmujący się przetwarzaniem po stronie przeglądarki. Każde zapytanie klienta jest weryfikowane za pomocą modułu \emph{ACL}\footnote{ang. Access Control List}, który sprawdza czy konfiguracja serwera zezwala na uruchamianie żądanych aplikacji. Szczegółowy opis modułu przedstawiono w sekcji \ref{sec:server-security}.

\begin{figure}[H]
\centering
\includegraphics[width=1.0\linewidth]{img/arch-www}
\caption{Schemat komunikacji z serwerem WWW.}
\label{fig:arch-www}
\end{figure}

\section{Komunikacja między klientem a aplikacją}

Do rozwiązania tego problemu konieczne jest utworzenie ciągłego kanału komunikacyjnego między klientem a procesem aplikacji, za pomocą którego będzie możliwe przesyłanie informacji o wyglądzie interfejsu aplikacji oraz informowanie aplikacji o zdarzeniach generowanych przez użytkownika po stronie przeglądarki. Jako, że za cel przyjęte zostało założenie o nieingerowaniu bezpośrednio w kod skompilowanych już aplikacji, głównym założeniem jest skorzystanie z techniki wstrzykiwania kodu biblioteki dynamicznej do przestrzeni pamięciowej procesu aplikacji tuż przed jego uruchomieniem. Kod ten ma za zadanie utrzymanie połączenia oraz transmisję danych między klientem a aplikacją. Dokładny sposób użycia tej techniki jest przedstawiony w sekcji \ref{sec:ldpreload}.

\begin{figure}[H]
\centering
\includegraphics[width=1.0\linewidth]{img/arch-socket}
\caption{Schemat komunikacji z modułem WebSocket.}
\label{fig:arch-socket}
\end{figure}

\begin{figure}[H]
\centering
\includegraphics[width=1.0\linewidth]{img/arch-hook}
\caption{Schemat powiązań modułu WebSocket z aplikacją \emph{Qt/KDE}.}
\label{fig:arch-hook}
\end{figure}

\section{Uzyskanie informacji o wyglądzie elementów graficznego interfejsu aplikacji}

Każdy element graficznego interfejsu aplikacji (\emph{QWidget}) jest renderowany w momencie odebrania zdarzenia \emph{QPaintEvent} z kolejki zdarzeń głównego wątku aplikacji. Dzięki temu istnieje łatwy sposób na uzyskanie informacji o tym kiedy oraz ktory element należy przerenderować aby uaktualnić jego wygląd po stronie klienta. Problemem w dalszyb ciągu pozostaje jednak sposób na uzyskanie informacji o samym wyglądzie. 

Proponowane rozwiązanie polega na zaimplementowaniu abstrakcyjnego urządzenia wyjściowego reprezentującego przeglądarkę WWW po stronie klienta (patrz podrozdział \ref{system_rysowania}). Odpowiednio implementując klasy \emph{QPaintEngine} oraz \emph{QPaintDevice} możliwe staje się uzyskanie szczegółowych informacji dotyczących wygądu widgetów co z kolei umozliwia stworzenie innowacyjnego formatu przesyłanych danych. Jego innowacyjność polega na przerzuceniu odpowiedzialności za rysowanie elementów piksel po pikselu na stronę klienta na podstawie dostarczonych przez serwer niezbędnych do tego celu informacji. Zamiast przesyłać bitmapy z wyrenderowanym elementem można wysłać informacje o kolorach, punktach, liniach i innych podstawowych elementach, które zostaną narysowane na urządzeniu docelowym jakim po stronie klienta jest przeglądarka WWW z obsługą elementów \emph{canvas}. Na tle istniejących rozwiązań jest to podejście dotąd niespotykane.


\section{Symulacja interakcji użytkownika z interfejsem aplikacji}
Interakcja użytkownika z aplikacją sprowadza się do obsługi następujących zdarzeń:
\begin{enumerate}
  \item ruch myszy nad elementem,
  \item wciśnięcie, zwolnienie oraz dwuklik przycisku myszy,
  \item zmiana położenia kółka myszy,
  \item wciśnięcie oraz zwolnienie klawiszy na klawiaturze,
  \item zmiana rozmiaru okna aplikacji poprzez przeciąganie jego krawędzi,
  \item zamknięcie, minimalizacja lub maksymalizacja okna aplikacji.
\end{enumerate}
Większość z wyżej wymienionych elementów jest obsługiwana jako zdarzenia w języku JavaScript większości dzisiejszych przeglądarek. Proponowane podejście na rozwiązanie tego zagadnienia polega na stworzeniu formatu danych bazując na notacji JSON (JavaScript Object Notation). Dane w tym formacie przesyłane do serwera są następnie poddawane walidacji i konwersji na obiekty zdarzeń biblioteki \emph{Qt}. Zdarzenia takie są następnie przesyłane do kolejki zdarzeń w głównym wątku aplikacji.

Odbiorcą zdarzenia jest obiekt, który na rysunku \ref{fig:arch-hook} został oznaczony jako Event Dispather. Moduł ten jest następnie odpowiedzialny za podjęcie akcji w odpowiedzi na dane zdarzenie, polegających na wywołaniu odpowiednich metod na elemencie interfejsu graficznego aplikacji, który wygenerował dane zdarzenie po stronie przeglądarki bazując na hierarchicznej budowie interfejsu użytkownika.
Wyjątkami są tutaj zdarzenia klawiatury, które nie mają bezpośredniego odbiorcy w momencie ich zaistnienia. Aplikacja na ogół sama decyduje o tym, który element powinien odebrać zdarzenie. Domyślnie jest to widget, który atualnie posiada tzw. focus, a to z kolei zależy od poprzednich zdarzeń oraz logiki samego programu. W celu symulacji podobnego zachowania moduł Event Dispather wybiera element interfejsu bazując na aktualnym stanie aplikacji i wysyła informację o wciśnięciu wirtualnych klawiszy. 

Taka struktura modułu obsługi zdarzeń zapewnia jednolitość działania aplikacji pomiędzy różnymi wersjami biblioteki Qt.

\chapter{Implementacja}
\section{Renderowanie}

\begin{figure}[H]
\centering
\includegraphics[width=0.8\linewidth]{img/arch-render}
\caption{Schemat architektury odtwarzania wyglądu aplikacji po stronie klienta.}
\label{fig:arch-render}
\end{figure}

\section{Zdarzenia}
\subsection{Zdarzenia myszy}
W zdarzeniach myszy używane są następujące pola:
\begin{itemize}
\item type -- typ zdarzenia,
\item id -- identyfikator widgeta, którego dotyczy zdarzenie,
\item x -- pozycja na osi X w momencie zajścia zdarzenia,
\item y -- pozycja na osi Y w momencie zajścia zdarzenia,
\item ox -- poprzednia pozycja na osi X, używane tylko przy ruchu,
\item oy -- poprzednia pozycja na osi Y, używane tylko przy ruchu,
\item btn -- wartość liczbowa określająca przyciski wciśnięte w momencie zajścia zdarzenia,
\item modifiers -- wartość liczbowa określająca klawisze specjalne (Alt, Control, Shift, klawisz Windows, klawisz Menu) wciśnięte w momencie zajścia zdarzenia,
\item delta -- wartość przesunięcia, używane tylko przy przewijaniu,
\item orientation -- kierunek, używane tylko przy przewijaniu.
\end{itemize}

\subsubsection{Ruch myszy}

\begin{lstlisting}[language=JavaScript,numbers=none]
{
  "command": "mouse",
  "type": "move",
  "id": 123456,  // Identyfikator obiektu, ktorego dotyczy zdarzenie
  "x": 0.0,
  "y": 0.0,
  "ox": 0.0,
  "oy": 0.0,
  "btn": 0x0,    // 0x00000000 Qt::NoButton
                 // 0x00000001 Qt::LeftButton
                 // 0x00000002 Qt::RightButton
                 // 0x00000004 Qt::MiddleButton
                 // 0x00000008 Qt::XButton1
                 // 0x00000010 Qt::XButton2
  "modifiers": 0x0  // 0x00000000 Qt::NoModifier
                    // 0x02000000 Qt::ShiftModifier
                    // 0x04000000 Qt::ControlModifier
                    // 0x08000000 Qt::AltModifier
                    // 0x10000000 Qt::MetaModifier
                    // 0x20000000 Qt::KeypadModifier
                    // 0x40000000 Qt::GroupSwitchModifier
}
\end{lstlisting}

\subsubsection{Wciśnięcie przycisku myszy}
\begin{lstlisting}[language=JavaScript,numbers=none]
{
  "command": "mouse",
  "type": "press",
  "id": 123456,
  "x": 0.0,
  "y": 0.0,
  "btn": 0x00000001,       // Qt::LeftButton
  "modifiers": 0x00000000  //  Qt::NoModifier
}
\end{lstlisting}

\subsubsection{Zwolnienie przycisku myszy}
\begin{lstlisting}[language=JavaScript,numbers=none]
{
  "command": "mouse",
  "type": "release",
  "id": 123456,
  "x": 0.0,
  "y": 0.0,
  "btn": 0x00000001,       // Qt::LeftButton
  "modifiers": 0x02000000  //  Qt::ShiftModifier
}
\end{lstlisting}

\subsubsection{Podwójne kliknięcie}
\begin{lstlisting}[language=JavaScript,numbers=none]
{
  "command": "mouse",
  "type": "dbclick",
  "id": 123456,
  "x": 0.0,
  "y": 0.0,
  "btn": 0x00000002,       // Qt::RightButton
  "modifiers": 0x00000000  //  Qt::NoModifier
}
\end{lstlisting}

\subsubsection{Zmiana położenia kółka myszy}
\begin{lstlisting}[language=JavaScript,numbers=none]
{
  "command": "wheel",
  "id": 123456,
  "x": 0.0,
  "y": 0.0,
  "btn": 0x00000002,       // Qt::RightButton
  "modifiers": 0x00000000  //  Qt::NoModifier
  "delta": 120,        // Zmiana polozenia
  "orientation": 0x    // 0x1 Qt::Horizontal
                       // 0x2 Qt::Vertical
}
\end{lstlisting} 

\subsection{Zdarzenia klawiatury}
\subsubsection{Wciśnięcie klawisza}
\begin{lstlisting}[language=JavaScript,numbers=none]
{
  "command": "key",
  "type": "press", 
  "key": 0x193,    // Kod klawisza
  "text": "a",     // Ciag znakow UTF8 bedacy rezultatem zdarzenia
  "autorep": true|false, // Okresla czy zdarzenie jest wynikiem
                         // przytrzymania klawisza przed dluzszy czas
  "count": 0,    // Okresla ile powtorzen klawisza mialo miejsce
  "modifiers": 0x0  // 0x00000000 Qt::NoModifier
                    // 0x02000000 Qt::ShiftModifier
                    // 0x04000000 Qt::ControlModifier
                    // 0x08000000 Qt::AltModifier
                    // 0x10000000 Qt::MetaModifier
                    // 0x20000000 Qt::KeypadModifier
                    // 0x40000000 Qt::GroupSwitchModifier
}
\end{lstlisting}

\subsubsection{Zwolnienie klawisza}
\begin{lstlisting}[language=JavaScript,numbers=none]
{
  "command": "key",
  "type": "release", 
  "key": 0x193,    // Kod klawisza
  "text": "a",     // Ciag znakow UTF8 bedacy rezultatem zdarzenia
  "autorep": true|false, // Okresla czy zdarzenie jest wynikiem
                         // przytrzymania klawisza przed dluzszy czas
  "count": 0,    // Okresla ile powtorzen klawisza mialo miejsce
  "modifiers": 0x0  // Qt::NoModifier
}
\end{lstlisting}
 

\subsection{Zmiana rozmiaru okna}
\input{implementation/events/resize.tex}

\subsection{Aktywacja okna}
\begin{lstlisting}[language=JavaScript,numbers=none]
{
  "command": "activate",
  "id": 123456   // Identyfikator obiektu, ktorego dotyczy zdarzenie
}
\end{lstlisting} 

\subsection{Zamknięcie okna}
\begin{lstlisting}[language=JavaScript,numbers=none]
{
  "command": "close",
  "id": 123456,   // Identyfikator obiektu, ktorego dotyczy zdarzenie
}
\end{lstlisting} 


\section{Szczegóły implementacji po stronie serwera}
\label{sec:implementation_server}
\subsection{Kolejka zdarzeń}
Aplikacje z graficznym interfejsem opartym o bibliotekę Qt wykorzystują kolejkę zdarzeń do asynchronicznej komunikacji zarówno między obiektami wewnątrz aplikacji jak i aplikacji z systemem operacyjnym. Wszystkie zdarzenia pochodzące od systemu operacyjnego są zamieniane na wewnętrzne zdarzenia biblioteki Qt aby mogły zostać przetworzone przez aplikację. I tak na przykład zwadzenie XMouse jest zamieniane na zdarzenie QMouseEvent, a XFocus na QFocusEvent. 

Podobny model zastosowany został w projekcie serwera aplikacji do komunikacji klienta z aplikacją uruchomioną na serwerze. Wszystkie zdarzenia generowane przez klienta za pomocą przeglądarki wysyłane są w przedstawionym w sekcji .... formacie. Z oczywistych względów dane takie należy przetworzyć aby mogły zostać przetworzone przez aplikację. Do tego celu stworzona została specjalna klasa zajmująca się parsowaniem i konwersją danych wejściowych na obiekty. 

Obiekty takie nie mogą być bezpośrednio przekazane aplikacji ponieważ część implementacji kolejki zdarzeń nie jest dostępna dla programistów ze względów bezpieczeństwa, w związku z tym nie jest możliwa np. kontrola fucusu na określone okno aplikacji tylko z wykorzystaniem zdarzeń, lecz kontrola taka musi odbywać się w sposób manualny poprzez wywoływanie odpowiednich metody biblioteki Qt.

W związku z powyższym stworzona została dodatkowa warstwa obsługi zdarzeń, która w reakcji na odbiór zdarzeń pośrednich podejmuje decyzję o przekazaniu zdarzenia bezpśrednio do samej aplikacji lub też przejmuje na siebie obowiązek sterowania zachowaniem aplikacji w odpowiedzi na dane zdarzenie poprzez uruchamianie odpowiednich metod i procedur. 

Taki podział strukturalny modelu zdarzeń umożliwia przede wszystkim łatwą jej modyfikację oraz dodawanie obsługi nowych zdarzeń w przyszłości (np. zdarzenia do obsługi urządzeń z ekranem dotykowym).

\subsection{Buforowanie grafik}
Większość współczesnych aplikacji z graficznym interfejsem użytkownika zawiera pewne elementy jak ikona lub obraz w formacie PNG (ang. Portable Network Graphics), których nie da się opisać za pomocą prostych elementów jak linia czy elipsa, lecz trzeba je przesłać w postaci danych binarnych. W przypadku takich aplikacji przy każdej zmianie widgeta zawierającego grafikę następuje ponowne jest narysowanie, a więc ta sama ikona czy obraz PNG będzie wykorzystywany wielokrotnie. 

Problem ten rozwiązano poprzez zastosowanie buforowania obrazów. Każdy taki element najpierw poddawany jest operacji wyznaczania wartości skrótowej (ang. hash value) z wykorzystaniem algorytmu MD5. Klucz ten przesyłany jest w klientowi zamiast właściwych danych obrazka, a sama grafika jest umieszczana w buforze serwera aplikacji. Klient chcąc narysować grafikę pobiera ją z serwera podając w zapytaniu otrzymany klucz. Po pobraniu obrazu serwer usuwa go z pamięci lecz nie zapomina o nim całkowicie. Zapamiętuje bowiem informację o tym, że dany obrazek został już przez klienta pobrany i od tej pory to klient jest odpowiedzialny za ponowne wykorzystanie pobranego wcześniej pliku. Serwer nie umieści ponownie w buforze obrazka o identycznej wartości skrótowej, dopóki klient nie poinformuje serwera, że nie przechowuje już dłużej danej grafiki i w przypadku jej ponownego wykorzystania serwer musi ponownie umieścić plik w buforze. 

Takie rozwiązanie zapewnia:
\begin{itemize}
\item zabezpieczenie przed zapełnieniem bufora serwera,
\item znaczne zmniejszenie transferu danych w aplikacjach wykorzystujących grafiki rastrowe --- przeprowadzone testy wykazały, że szybkość działania aplikacji wzrasta nawet dwukrotnie,
\item możliwość kontroli wielkości bufora przez klienta dzięki możliwości zdalnego czyszczenia pamięci kluczy.
\end{itemize}

\section{Szczegóły implementacji po stronie klienta}

\subsection{Biblioteki pomocniczne}
Podczas pracy z projektem zostały wykorzystane popularne biblioteki ułatwiające podstawowe zadania. Wszystkie z nich są darmowe nawet w wykorzystaniu komercyjnym oraz całkowicie otwarte.
\begin{enumerate}
  \item jQuery - biblioteka ułatwijąca operacje na elementach DOM
  \item jQuery-UI - plugin do biblioteki jQuery umożliwiający tworzenie zaawansowanych wizualnych elementów, takich jak okna dialogowe, elementy rozszerzalne (ang. resizable), elementy przesuwalne (ang. draggable).
  \item jQuery-mousewheel - plugin do biblioteki jQuery dodający obsługę kółka myszy
  \item sylvester.js - biblioteka służąca do zaawansowanych obliczeń na macierzach i wektorach
\end{enumerate}


\subsection{Połączenie}
Klient po załadowaniu początkowej strony dokonuje połączenia WebSocket z serwerem na port podany na stronie. Następnie nasłuchuje na informacje od serwera i na każdą z nich reaguje. Komunikaty w formacie JSON w drugą stronę wysyłane są po zajściu zdarzeń po stronie klienta, np. ruchu myszą. Dokładny opis zdarzeń znajduję się w sekcji % TODO

\subsection{Window manager (ang. zarządca okien) na szybkości}
Typowe programy komputerowe składaja się z wielu okien. Sposób implementacji pseudookien i zarządcy w projekcie był możliwy na dwa sposoby.
Pierwszym wariantem jest zastosowanie osobnych okien przeglądarki (tzw. popupów), które odpowiadałyby rzeczywistym oknom przeglądarki.
Drugą opcją jest stworzenie minimalistycznego managera okien w języku Javascript.

Minusem pierwszym rozwiązania jest całkowitym brak możliwości sterowania oknami. Przeglądarka nie jest w stanie zablokować możliwości zamknięcia okna, sterować ich modalnościa oraz przyciskami sterowania (minimalizacji, maksymalizacji, zamknięcia i innymi). Co więcej stosowanie dodatkowych okien przeglądarki jest uważane za złą praktykę.
Z powodu tak dużej ilości problemów w projekcie zdecydowano się na użycie drugiego wariantu. Do stworzenia okien wykorzystana została biblioteka jQuery-UI dostarczająca metody umożliwiający w przystępny sposób tworzenie elementów rozszerzalnych (ang. resizable) oraz przeciągalnych (ang. draggable). Celem funkcjonalnym było upodobnienie zachowania pseudookien w przeglądarce do prawdziwych okien programu:
\begin{enumerate}
  \item zmiana rozmiaru okna przy pomocy uchwytów w rogach i na krawędziach
  \item przenoszenie okien za pomocą paska tytułowego
  \item minimalizacja, maksymalizacja oraz zamykanie okien za pomocą przycisków w prawym górnym rogu
\end{enumerate}

%* TODO - tutaj rysunk okienka z zaznaczonymi elementami typu pasek Bar, przyciski _,[],X, chywataki resizable.

Wszystkie elementy są prostymi obiektami blokowymi typu div. Odpowiednie usytułowanie elementów zapewniają właściwości CSS.
Funkcjonalność maksymalizacji różni się od funkcjonalności w rzeczywistym środowisku. Jest to równoważne ze zwiększeniem rozmiaru okna do maksymalnego dostępnu obszaru (100 procent szerokości i wysokości ciała strony).

Taka implementacja problemu wynika z możliwości uruchomienia serwera aplikacji w serwerze X w dowolnej rozdzielczości. Po faktycznym zmaksymalizowaniu okna na serwerze użytkownik po stronie przeglądarki widziałby okno o rozmiarze innym niż pełny dostępny obszar widoku. W przypadku rozmiaru mniejszego skutkowałoby to pustym, niezagospodarowanym miejscem w oknie przeglądarki, natomiast większy rozmiar powodowałby pojawienie się suwaków przewijania (ang. scrollbar).

Funkcjonalność minimalizacji okna jest również symulowana. Okno jest chowane poprzez zmianę wartości CSS display na none. Dodatkowo tworzony jest element na pasku zadań, który po kliknięciu przywraca ukryte okno. Pasek zadań jest elementem strony znajdującym się na samym dole.

** TODO rysunek paska zadań **

\subsection{Rysowanie pojedynczego widgeta}
Na stronie widget reprezentowany jest przez element kontenera div oraz zawarty w nim element canvas oraz kontener na dzieci widgeta. Element kontenera ma jedynie funkcję pomocniczą i rozmieszcza w odpowiedniej pozycji canvas oraz dzieci. Jest całkowicie przezroczysty i nie jest widoczny na stronie. Taka implementacja relacji rodzic-dziecko widgetów na stronie zapewnia drzewiastość i łatwość zarządzania widgetami. Zmiana rodzica widgeta, który posiada zagnieżdzone dzieci nie jest problematyczna, a ze względu na format przesyłanych danych od serwera ta operacja jest bardzo często używana na etapie tworzenia okien. Najstarsze widgety (bezpośrednio dziedziczące po QDialog) stoją najwyżej w strukturze i są dodatkowo opakowane w kontener okna z całą jego strukturą.

** TODO rysunek struktury **

\section{Format danych}

\subsection{Dane wysyłane przez serwera}

** Trójki - QT | JSON | canvas **

\subsection{Dane wysyłane przez klienta}

\section{Napotkane problemy}
\begin{enumerate}
  \item Rysowanie za pomocą zdarzeń, synchronizacja i buforowanie
  \label{rendering_events}
  \item Znikający \emph{focus} okna aplikacji
  \label{problems_focus}
  \item Kody znaków klawiatury
  \label{problems_keyboard}
  \item ....
  \item problem z clippingiem w przeglądarkach
\end{enumerate}

Ad.\ref{rendering_events} Rysowanie widgetów we frameworku \emph{Qt} realizowane jest wewnątrz kolejki zdarzeń aplikacji. Zdarzenie rysowania powiadamia element interfejsu o konieczności przerysowania. Widget posiada wskaźnik do miejsca w pamięci gdzie powinien przeprowadzic operację renderowania, gdyż sam jest implementacją klasy QPaintDevice. 

Powyższy schemat działania wymagał znalezienia sposobu na zmuszenie widgetów do rysowania za pomocą specjalnie przygotowanej implementacji klasy QPaintEngine oraz QPaintDevice. Do tego celu wykorzystana została metoda biblioteki Qt:

\begin{lstlisting}[language=C++,numbers=none]
void QWidget::render(QPainter * painter, 
                     const QPoint & targetOffset = QPoint(), 
                     const QRegion & sourceRegion = QRegion(), 
                     RenderFlags renderFlags 
                          = RenderFlags(DrawWindowBackground | 
                                        DrawChildren))
\end{lstlisting}

Umożliwiła nam ona wskazanie obiektu QPainter wykorzystującego mechanizm renderowania serwera, tj. reimplementacje klasy QPaintEngine oraz QPaintDevice. Wykorzystanie tej metody powoduje wygenerowanie kolejnego zdarzenia i umieszczenie go w kolejce aplikacji. Powodowało to problem wpadania serwera w nieskończoną pętlę i uniemożliwiało jego dalsze poprawne funkcjonowanie. 

Rozwiązaniem problemu było stworzenie własnej kolejki widgetów, które wymagają renderowania. Kolejka ta jest opróżniana w pewnych odstępach czasu nie krótszych niż 100 milisekund. Wartość ta została dobrana eksperymentalnie tak aby uzyskać efekt płynnej interakcji z aplikacją.

Ad.\ref{problems_focus} Biblioteka \emph{Qt} stanowi niejako nakładkę dla natywnego zarządcy okien systemu operacyjnego. W zwizku z tym o kolejności okien na stosie decyduje system operacyjny. \emph{Qt} podejmuje jedynie odpowiednie czynności w celu aktualizacji graficznego interfejsu aplikacji w zależności od aktualnego stanu konkretnych okien definiowanego przez system operacyjny. W sytuacji kiedy użytkownik nie prowadzi żadnej intrakcji z systemem operacyjnym a jedynie z aplikacją, mogło by się zdarzyć, że niektóe zdarzenia było by ignorowane przez aplikację. Przykładem takiej sytuacji jest wprowadzanie tekstu na klawiaturze. \emph{Qt} przesyła zdarzenia klawiatury do widgeta, który aktualnie posiada focus w oknie, które znajduje się na szczycie stosu okien w systemie operacyjnym. W momencie kiedy dwóch zdalnych użytkowników uruchomiło by aplikację na tym samym serwerze, zdarzenia jednego użytkownika były by ignorowane ponieważ tylko jedno okno jednej aplikacji może być na szczycie stosu. 

Rozwiązaniem tego problemu była reimplementacja odpowiednich metod \emph{Qt} w celu symulacji zachowania stosu okien systemu operacyjnego wewnątrz samej aplikacji. W rezultacie aplikacja działa tak jakby zawsze była aktywna, dzięki czemu framework \emph{Qt} poprawnie renderuje wszystkie elementy interfejsu użytkownika.

Ad.\ref{problems_keyboard} 
\emph{Qt} dostarcza platformowo niezależny opis kodów znaków klawiatury za pomocą zdefiniowanego typu wyliczeniowego \emph{Qt::Key}\footnote{http://doc.qt.digia.com/qt/qt.html\#Key-enum}. Niestety przeglądarki nie są dobrze ustandaryzowane i ich numeracja klawiszy znacznie różni się nie tylko między samymi platformami ale również między ich wersjami. Dodatkowo przeglądarki często nie wspierają wszystkich klawiszy przez to zakres kodów znaków jest inny niż w przypadku biblioteki \emph{Qt}.

Powyższe problemy wymusiły konieczność zaimplementowania metody konwertującej po stronie klienta kody klawiszy na odpowiadające im wartości typu wyliczeniowego \emph{Qt::Key}.

|| TODO: Jak to Janek zrobił


\chapter{Testy aplikacji}
Podstawowym problemem w użytkowaniu zdalnych aplikacji niezależnie od protokołu jest ilość przesłanych danych oraz opóźnienie. Przeprowadzono proste testy porównujące pod tym względem stworzone rozwiązanie z systemem \emph{VNC}. Porównanie to nie jest bardzo dokładne i przedstawione dane są jedynie poglądowe. Pełna analiza wymaga dokładnego poznania protokołu \emph{RFB}, który wykorzystuje \emph{VNC} i modyfikacji obecnych klientów w celu zebrania danych diagnostycznych.

Specyfikacja sprzętowa.
\begin{itemize}
\item Klient oraz maszyna lokalna -- procesor Intel Core i5-2400, 8GB pamięci RAM, dysk SSD.
\item Serwer zdalny -- VPS (ang. Virtual Private Server), procesor Intel Core i5-3570K z dostępem do 2 rdzeni, 2GB pamięci RAM, współdzielony dysk SSD.
\end{itemize}

Specyfikacja oprogramowania.
\begin{itemize}
\item Klient oraz maszyna lokalna --- openSUSE Linux 12.1, Qt 4.7.4, KDE 4.7.2, klient VNC TightVNC Viewer 1.3.10, Mozilla Firefox 17.0.1.
\item Serwer zdalny --- Debian Unstable, Qt 4.8.2, KDE 4.8.4, serwer VNC TightVNC Server 1.3.9.
\end{itemize}

Na obu komputerach oprogramowanie zostało zainstalowane w standardowy sposób oraz nie dokonywano żadnych zmian w ustawieniach. Używany jest domyślny styl \emph{Oxygen}.

Do porównania wybrano dwa programy -- kalkulator \emph{KCalc} dostępny w podstawowym pakiecie KDE oraz grę pasjans \emph{KPat} (aka \emph{KPatience}) z pakietu KDE Games. \emph{KCalc} jest aplikacją wykorzystującą podstawowe widgety, gdzie praktycznie nie używane są bitmapy. Gra \emph{KPat} prawie w całości bazuje na elemencie, który w \emph{Qt} renderowany jest za pomocą bitmap. Wybór tych programów pozwoli na porównanie działania dla dwóch różnych przypadków.

\section {Testy funkcjonalne}
Z powodu poziomu skomplikowania aplikacji testy funkcjonalne przeprowadzono ręcznie. Polegały na uruchomieniu wielu programów użytkowych z pakietu \emph{KDE} takich jak:
\begin{itemize}
\item \emph{KCalc},
\item \emph{Dolphin},
\item \emph{QtCreator},
\item \emph{KPat},
\item \emph{Kwrite},
\item \emph{Konsole}.
\end{itemize}

Aplikacje były sprawdzane pod względem poprawności wyświetlania elementów, funkcjonalności menadżera okien i możliwości interakcji użytkownika. Przykładowe aplikacje uruchomione przy pomocy projektu przedstawione są na rysunkach \ref{dolphin}, \ref{qtcreator} oraz \ref{konsole}.

\begin{figure}[!h]
  \centering
  \includegraphics[width=\textwidth,height=!]{img/dolphin.png}
  \caption{Aplikacja \emph{Dolphin}}
  \label{dolphin}
\end{figure}

\begin{figure}[!h]
  \centering
  \includegraphics[width=\textwidth,height=!]{img/qtcreator.png}
  \caption{Aplikacja \emph{QtCreator}}
  \label{qtcreator}
\end{figure}

\begin{figure}[!h]
  \centering
  \includegraphics[width=\textwidth,height=!]{img/konsole.png}
  \caption{Aplikacja \emph{Konsole}}
  \label{konsole}
\end{figure}

\section{Testy wydajnościowe}

\subsection{Badanie ilości przesłanych danych}

Pierwszym testem jest sprawdzenie ilości przesłanych. Do pomiaru użyto program \emph{iftop}, wykonano 5 prób, wyniki podane w kilobajtach.
Stworzono dwa przypadki testowe wykonane w 5 próbach. Pierwszy polega na włączneniu aplikacji i pomiarze przesłanych danych aż do pełnego wyświetlenia się na ekranie. Wyniki znajdują się w tablicach \ref{tab:test1} oraz \ref{tab:test2}. Drugim scenariuszem jest sprawdzenie działania aplikacji w dłuższym okresie czasu. Dokonano nagrania prostego makra z użyciem myszy za pomocą aplikacji \emph{xmacro2}. Testu dokonywano przez około minutę testując cykliczne kliknięcia na wybranych elementach w stałych odstępach czasu. Wyniki dostępne w tablicach \ref{tab:test3} oraz \ref{tab:test4}.

Wyniki wskazują wyższość systemu \emph{VNC} pod kątem ilości przesłanych danych. Różnice te wynikają przede wszystkim z narzutu protokołu. \emph{RFB} jest protokołem binarnym, gdzie przesyłane są głównie obrazy. W protokole użytym w projekcie duży narzut stanowi użyty format \emph{JSON}. Obejściem tego problemu może być próba zastosowanie kompresji \emph{GZIP}. Przesyłane dane ze względu na ogarniczoną ilość użytych znaków, które często się powtarzają bardzo dobrze się kompersują -- uzyskiwany jest kilkukrotny zysk objętościowym. Za pomocą \emph{WebWorkerów} HTML5 możliwa jest dekompresja w wielu wątkach w celu przyspieszenia aplikacji.

\begin{table}
\centering  
\begin{tabular}{l*{6}{|c}r}
\multicolumn{7}{c}{KCalc} \\
\hline
Aplikacja              & 1 & 2 & 3 & 4 & 5  & Średnia  \\
\hline
KAppstream & 560 & 1002 & 494 & 506 & 641 &  640  \\
\hline
VNC            & 592 & 621 & 651 & 575 &  584 &  604  \\
\end{tabular}
\caption{Test aplikacji KCalc -- ilość przesłanych danych od startu}
\label{tab:test1}
\end{table}

\begin{table}
\centering  
\begin{tabular}{l*{6}{|c}r}
\multicolumn{7}{c}{KPat} \\
\hline
Aplikacja              & 1 & 2 & 3 & 4 & 5  & Średnia  \\
\hline
KAppstream & 672 & 642 & 694 & 705 & 650 &  672  \\
\hline
VNC            & 619 & 981 & 651 & 575 &  610 &  687  \\
\end{tabular}
\caption{Test aplikacji KPat -- ilość przesłanych danych od startu}
\label{tab:test2}
\end{table}


\begin{table}
\centering  
\begin{tabular}{l*{6}{|c}r}
\multicolumn{7}{c}{KCalc} \\
\hline
Aplikacja              & 1 & 2 & 3 & 4 & 5  & Średnia  \\
\hline
KAppstream & 1802 & 2021 & 1902 & 2104 & 1802 &  1926  \\
\hline
VNC            & 1021 & 989 & 1101 & 1021 &  1258 &  1078  \\
\end{tabular}
\caption{Test działania aplikacji KCalc  -- symulacja działania użytkownika}
\label{tab:test3}
\end{table}

\begin{table}
\centering
\begin{tabular}{l*{6}{|c}r}
\multicolumn{7}{c}{KPat} \\
\hline
Aplikacja              & 1 & 2 & 3 & 4 & 5  & Średnia  \\
\hline
KAppstream & 10847 & 9054 & 10814 & 9581 & 10485 & 10156   \\
\hline
VNC            & 6945 & 7541 & 6801 & 6541 &  7211 & 7007   \\
\end{tabular}
\caption{Test działania aplikacji KPat -- symulacja działania użytkownika}
\label{tab:test4}
\end{table}

\subsection{Badanie działania przy słabym połączeniu i dużych opóźnieniach}

Do symulacji pracy przy użyciu słabego łącza wykorzystano pakiet netem \cite{netem}.


\section{Podsumowanie testów}
W środowisku lokalnym oba rozwiązania praktycznie nie różnią się od siebie w kwestii wydajności. Przewaga systemu \emph{VNC} w sieci Internet wynika głównie z powodu mniejszej ilości przesyłanych danych. Przy zastosowaniu dodatkowych rozwiązań kompresujących dane możliwe będzie uzyskanie wyniku zbliżonego, lub nawet niższego w szczególnych przypadkach.
Nowoczesne przeglądarki bardzo dobrze radzą sobie z rysowaniem. Dla testowanej przeglądarki \emph{Mozilla Firefox} czas rysowania na elemencie \emph{canvas} jest wartością całkowicie pomijalną. Rysowanie jednego widgeta trwa około 1-3 milisekund, w zależności od poziomu skomplikowania. Pomiarów tych dokonano przy pomocy metody \emph{Date.getTime()}.

\section{Testy w środowisku lokalnym}
\input{tests/local.tex}
Wszystko spoko.

\section{Testy w sieci Internet}
Wszystko wyszło gitara. Jesteśmy z siebie dumni. Pozdrawiamy mamę, tatę i dudniącego Krzysztofa.


\chapter{Podsumowanie}
Temat pracy inżynierskiej został w pełni zrealizowany, a jej wynikiem jest prototyp serwera oraz webowej aplikacji klienckiej. 

Program serwera udostępnia kod strony \emph{WWW}, który uruchamiany jest przez przeglądarkę. Uruchamia on również aplikację graficzną opartą o framework \emph{Qt} oraz odpowiada za utworzenie połączenia między klientem a procesem aplikacji. Możliwa jest także obsługa wielu połączeń równocześnie. 
Przy pomocy techniki wstrzykiwania kodu bibliotek linkowanych dynamicznie \emph{(ang. DLL injection)} oraz wewnętrznych mechanizmów biblioteki \emph{Qt}, użycie aplikacji nie wymaga ponownej kompilacji, zarówno samego frameworka \emph{Qt}, jak i uruchamianych aplikacji użytkowych.

Aplikacja kliencka stworzona w postaci dynamicznej strony \emph{WWW} pokrywa bardzo duży podzbiór funkcjonalności modułu graficznych interfejsów biblioteki \emph{Qt}. Wszelkie braki wynikają z niedoskonałości standardu \emph{HTML5}, który wciąż jest mocno rozwijany.

Projekt będzie kontynuowany w następujących kierunkach:
\begin{itemize}
  \item umożliwienie współpracy z aplikacjami opartymi o najnowszą bibliotekę \emph{Qt} w wersji 5.0,
  \item rozwinięcie zabezpieczeń --- autentykacja i autoryzacja klientów,
  \item stworzenie panelu administracyjnego serwera oraz nowego widoku głównego aplikacji,
  \item automatyzacja procesu instalacji,
  \item utworzenie wersji serwera dla systemów Windows oraz Mac OS,
  \item rozwinięcie możliwości aplikacji klienckiej przy wykorzystaniu elementów technologii \emph{HTML5}, które będą dostępne w przyszłości.
\end{itemize}

Stworzona implementacja jest w pełni otwarta --- kod źródłowy dostępny jest publicznie pod adresem \url{https://gitorious.org/kde-appstream/kde-appstream}.


%+Make Index
\printindex
%-Make Index

\end{document}